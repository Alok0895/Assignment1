\documentclass[letterpaper,12pt]{article}
\usepackage{tabularx} % extra features for tabular environment
\usepackage{amsmath}  % improve math presentation
\usepackage{graphicx} % takes care of graphic including machinery

\usepackage[margin=1in,letterpaper]{geometry} % decreases margins
\usepackage{cite} % takes care of citations
\usepackage[final]{hyperref} % adds hyper links inside the generated pdf file
\hypersetup{
	colorlinks=true,       % false: boxed links; true: colored links
	linkcolor=blue,        % color of internal links
	citecolor=blue,        % color of links to bibliography
	filecolor=magenta,     % color of file links
	urlcolor=blue         
}
\usepackage{blindtext}
%++++++++++++++++++++++++++++++++++++++++


\begin{document}

\title{MATRIX THEORY}
\author{ALOK RANJAN}
\date{\today}
\maketitle

\section{Question 43, P.59}
\subsection{CODE:-}
\begin{verbatim}
def print_value(x):
  
  #slope of line @ Origin
  m_origin=x[1]/x[0]
  
  #slope of the required line 
  m_line= -1/m_origin
  
  #required constant of the line
  c_line =x[1]-m_line*x[0]
  
  #printing the value of line slope and the constant
  print("the slope of m in the line is",m_line)
  print("the value of c in the line is", c_line)

x=[-1, 2]
print_value(x)
\end{verbatim}

\subsection{Explanation}
The line through the origin perpendicular to the given line is in the form of $ y=\hat{m}x$. 
Since this line passes through [-1,2]. Therefore, \\
\begin{equation}
    \hat{m}= \frac{y}{x} = \frac{2}{-1} = -2
\end{equation}
 Therefore, the slope of the required line
\begin{align}
\hat{m}m &= -1\notag\\
\implies m &= \frac{-1}{\hat{m}}\notag\\
\implies m &=0.5
\end{align}
The required constant value in the line is given by
\begin{align}
    c &= y-mx\notag\\
    \implies c &= 2-0.5x(-1)=2.5  \notag\\
    \implies c &=2.5
\end{align}

Hence, the value of m and c is obtained from (2) and (3) as 0.5 and 2.5 respectively.

    





\end{document}
