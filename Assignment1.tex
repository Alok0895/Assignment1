\documentclass[letterpaper,12pt]{article}
\usepackage{tabularx} % extra features for tabular environment
\usepackage{amsmath}  % improve math presentation
\usepackage{graphicx} % takes care of graphic including machinery

\usepackage[margin=1in,letterpaper]{geometry} % decreases margins
\usepackage{cite} % takes care of citations
\usepackage[final]{hyperref} % adds hyper links inside the generated pdf file
\hypersetup{
	colorlinks=true,       % false: boxed links; true: colored links
	linkcolor=blue,        % color of internal links
	citecolor=blue,        % color of links to bibliography
	filecolor=magenta,     % color of file links
	urlcolor=blue         
}
\usepackage{blindtext}
%++++++++++++++++++++++++++++++++++++++++
\begin{document}

\title{MATRIX THEORY}
\author{ALOK RANJAN}
\date{\today}
\maketitle

\section{Question 43, P.59}
The perpendicular from the origin to the line\\
(-m 1)x=c\\
meets it at the point (-1 2). Find the values of m and c.

\subsection{Python Code $\&$ Latex Code Links }
\fbox{\parbox{\textwidth}{
	1.Python Code:-\\
	https://github.com/Alok0895/Assignment1/blob/master/Assignment1.py}\\}\\
\fbox{\parbox{\textwidth}{2.Latex Code:-\\
https://github.com/Alok0895/Assignment1/blob/master/Assignment1.tex}}\\


\subsection{Explanation}
The line through the origin perpendicular to the given line is in the form of:-\\ 
$L_{1}\implies\hspace{10}(-m\hspace{5} 1)x=c$\\ Direction vector of line $L_{2}$ perpendicular to $L_{1}$ should be ($1\atop m$) such that the equation of line should be\\
$L_{2}\implies\hspace{10} ({1\atop m})^{T}\left(x-\left({-1\atop2}\right)\right)=0$\\
Therefore $L_{2}$ is,\\
\begin{equation}
\implies (1\hspace{5}m)x-(1\hspace{5}m)\left({-1\atop2}\right)=0    
\end{equation}

Since it passes through (0,0) therefore:-\\
\begin{align}
(1\hspace{5}m)\left({0\atop0}\right)-(1\hspace{5}m)\left({-1\atop2}\right)&=0\\  
\implies(-1+2m) &= 0\notag\\
\implies m &= \frac{1}{2}\notag\\
\implies m &=0.5
\end{align}
Now from equation(3) and the knowledge that $(-1,2)$ lies on L{}, we put value of m in line $L_{1}$ such that:-
\begin{align}
    (-0.5\hspace{5} 1)x &= c\\
    \implies(-0.5\hspace{5} 1)\left({-1\atop2}\right)&=c\notag\\
    \implies c&=2.5
\end{align}

Hence, the value of m and c are obtained from (2) and (3) as 0.5 and 2.5 respectively.\\

\begin{figure}
    \centering
    \includegraphics[width=18cm]{alok_plot.PNG}
    \caption{Lines perpendicular at (-1,2)}
    \label{fig:my_l}  
\end{figure}
    





\end{document}
